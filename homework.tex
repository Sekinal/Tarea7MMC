\documentclass{modernhandout}
\lhead{\small\sffamily\bfseries Irving Ernesto Quezada Ramírez}
\rhead{\small\sffamily\bfseries Mecánica del medio continuo}

\begin{document}

\section*{Problemas}
\begin{enumerate}
    \item Encuentre las componentes covariantes y contravariantes de un tensor en las siguientes coordenadas, si sus componentes rectangulares son $2x-z,\;x^2y,\;yz$.
    \begin{enumerate}
        \item Esféricas $r,\theta,\phi$
        \item Cilíndricas
    \end{enumerate}
    \item Las componentes contravariantes de un tensor en coordenadas rectangulares son $yz,\;3,\;2x+y$. Encuentre las componentes covariantes en coordenadas parabólicas cilíndricas.
    \item Suponga que $A^{pq}$ y $B_{rs}$ son tensores simétricos oblicuos. Demuestre que $C_{rs}^{pq}=A^{pq}B_{rs}$.
    \item Determine el tensor métrico y el tensor métrico conjugado en coordenadas cilíndricas elípticas.
    \item Determine $g$ y $g^{ij}$ que corresponden a $\mathrm{d}s^2=3\left(\mathrm{d}x^1\right)^2+2\left(\mathrm{d}x^2\right)^2+4\left(\mathrm{d}x^3\right)^2-6\left(\mathrm{d}x^1\mathrm{d}x^3\right)$.
    \item Exprese la relación entre los tensores asociados
    \begin{enumerate}
        \item $A^{pq}$ y $A^{\cdot q}_{j}$
        \item $A^{p\cdot r}_{\cdot q}$ y $A_{jgl}$
        \item $A^{\cdot\cdot r}_{pq}$ y $A^{jk}_{\cdot\cdot l}$
    \end{enumerate}
    \item Determine los símbolos de Christoffel de primer y segundo orden para coordenadas cilíndricas elípticas.
    \item Escriba la derivada covariante con respecto de $x^q$ de cada uno de los siguientes tensores:
    \begin{enumerate}
        \item $A^{jk}_{l}$
        \item $A^{jk}_{lm}$
        \item $A^{j}_{klm}$
        \item $A^{jkl}_{m}$
        \item $A^{jk}_{lmn}$
    \end{enumerate}
    \item Exprese la divergencia de un vector $A^p$ em términos de sus componentes físicas para coordenadas:
    \begin{enumerate}
        \item Cilíndricas parabólicas.
        \item Paraboloides.
    \end{enumerate}
    \item Determine $\nabla^2\Phi$ en coordenadas parabólicas.
    \item Con el empleo de notación tensorial, demuestre que
    \begin{enumerate}
        \item $\nabla\cdot\left(\nabla\times A^r\right)=0$
        \item $\nabla\times\left(\nabla\Phi\right)=0$
    \end{enumerate}
    \item La ecuación de continuidad está dada por $\nabla\cdot\left(\sigma\nu\right)+\frac{\partial\sigma}{\partial t}=0$, donde $\sigma$ es la densidad y $\nu$ es la velocidad del fluido. Exprese la ecuación de forma tensorial.
    \item Exprese la ecuación de continuidad en coordenadas cilíndricas.
    \item Exprese el teorema de Stokes en forma tensorial.
\end{enumerate}
\end{document}